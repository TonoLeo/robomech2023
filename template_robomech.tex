\documentclass{jarticle}
\usepackage{robomech}
\usepackage{graphicx}

\usepackage{hyperref}

\begin{document}
\makeatletter
\title{価値反復を用いた移動ロボットによる屋外ナビゲーション}
{―日本語副題:ゴシック体・12pt(欧文Arial・12pt)―}
{English Title: Times New Roman, 12pt}
{-English Subtitle: Times New Roman, 10pt-}

\author{
	\begin{tabular}{ll}
		○学\hspace{1zw}登内 リオン(千葉工大)& 正\hspace{1zw}林原 靖男\hspace{1zw} (千葉工大)\\
 		\hspace{1zw}正\hspace{1zw}上田 隆一(千葉工大)\\
		% ※協賛・後援団体の会員資格で発表される場合は「正・学」は不要です。
	\end{tabular}
	% &\\
	\vspace{1zh} \\
	\begin{tabular}{l}
			{\small Leon TONOUCHI, Chiba Institute of Technology, s20c1078un@s.chibakoudai.jp} \\
			{\small Ryuichi UEDA, Chiba Institute of Technology} \\
			{\small Yasuo HAYASHIBARA, Chiba Institute of Technology}             \\
	\end{tabular}
}
\makeatother

\abstract{ \small
	Papers submitted must be original, and previously unpublished. The responsibility for the contents of published articles rests solely with the authors and not the society. Copyright of the papers published belongs to the JSME (Japan Society of Mechanical Engineers). [Abstract: Times New Roman, 9pt, 100-150words]
}

\date{} % 日付を出力しない
\keywords{Robot, Manipulation,… (no more than five words) [Times New Roman, 9pt]}

\maketitle
\thispagestyle{empty}
\pagestyle{empty}

\small
\section{緒言}%===========================
%本文:明朝体・9pt(欧文Times New Roman, 9pt)、文字間隔は1行26文字程度、行間隔は4.2mm程度にして下さい。
自律移動ロボットが実環境で走行することを考えると,変化する目的地や経路に対して即座に最適な行動計画が必要となる.価値反復を移動ロボットのナビゲーションに適用するとロボットの様々な状態に対して最適な行動を計算することができる.
%\subsection{論文作成に関する注意事項を以下に示します。(中見出し:ゴシック体・9pt・強調文字・左寄せ)}%-----------


%※ ただし、PDFファイルの容量は2MB以下、論文のページ数は2頁以上4頁以下とします。なお、印刷原稿の提出は不要ですので、郵送しないで下さい。

%※ 講演番号、講演会名、ページ番号は記載しないようにして下さい。
\section{実験}%===========================
本研究では上田らが開発したROSパッケージ(\href{https://github.com/ryuichiueda/value_iteration}{https://github.com/ryuichiueda/value\_iteration})[]を実装し実験を行った.
このパッケージを用いると100$\mathrm{[m^2]}$の広さの環境で2秒の計算コストで行動計画が可能である.
そのため今回は千葉工業大学津田沼キャンパス内でロボット(Raspberry Pi Cat)を走行させ計算量を測定する.
地図のスタート地点から走行させ,各ポイントを通過するコースを設定し,価値反復の計算時間を計測した.
計算機には CPU として Intel Core i7-12700H(14コア28スレッド), DRAM として DDR4-3200 16GB を用いた.
自己位置推定には上田らが作成したROSのemclパッケージ(\href{https://github.com/ryuichiueda/emcl2}{https://github.com/ryuichiueda/emcl2})[],価値反復の際のコストの計算には占有格子地図を用いた.

Table 1 には,地図の大きさ,格子の解像度,ロボットが移動できる地図のセルの数,面積を記載する.
%地図の画像とポイントを記載した画像はこれから

\begin{table}[hbtp]
  \caption{configurations of the map}
  %\label{table:data_type}
  \centering
  \begin{tabular}{l|r}
    \hline
    map size & \\
    cell resolution &  \\
		number of cells & \\
    number of free cells & \\
		the area of the free cells & \\
    \hline
  \end{tabular}
\end{table}

\begin{table}[hbtp]
  \caption{parameters for value iterations}
  %\label{table:data_type}
  \centering
  \begin{tabular}{l|r}
    \hline
    number of states & \\
    number of free states &  \\
		number of actions & \\
    \hline
  \end{tabular}
\end{table}

\subsection{計算量の測定}

\subsection{実験結果}

\section{結論}%===========================




\footnotesize
\begin{thebibliography}{99}

	\bibitem{Shinjuku1}
	Bellman, R., ``{\it Dynamic Programming},'' Princeton Uni-versity Press, Princeton, NJ, 1957.

	\bibitem{Shinjuku2}
	上田隆一,池邉龍宏,林原靖男,``brute-forceな価値反復を用いた実時間経路計画ROSパッケージ'', 
	第39回日本ロボット学会学術講演会予稿集, 2021.

	\bibitem{Shinjuku3}
	上田隆一,池邉龍宏,林原靖男,``移動ロボットのナビゲーションのためのbrute-forceな価値反復を用いた大域計画・局所計画アルゴリズム'', 
	第27回ロボティクスシンポジア講演論文集, 2022.

	\bibitem{Shinjuku4}
	Ueda, R., {\it et al}., ``Real-Time Decision Making with State-Value Function under Uncertainty of State Estimation,''
	 in {\it Proc. of} ICRA, 2005.

\end{thebibliography}

\normalsize
\end{document}
