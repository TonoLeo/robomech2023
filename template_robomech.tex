\documentclass{jarticle}
\usepackage{robomech}
\usepackage{graphicx}

\begin{document}
\makeatletter
\title{価値反復を用いた移動ロボットによる屋外ナビゲーション}
{―日本語副題:ゴシック体・12pt(欧文Arial・12pt)―}
{English Title: Times New Roman, 12pt}
{-English Subtitle: Times New Roman, 10pt-}

\author{
	\begin{tabular}{ll}
		○学\hspace{1zw}登内 リオン(千葉工大)& 正\hspace{1zw}林原 靖男\hspace{1zw} (千葉工大)\\
 		\hspace{1zw}正\hspace{1zw}上田 隆一(千葉工大)\\
		% ※協賛・後援団体の会員資格で発表される場合は「正・学」は不要です。
	\end{tabular}
	% &\\
	\vspace{1zh} \\
	\begin{tabular}{l}
			{\small Leon TONOUCHI, Chiba Institute of Technology, s20c1078un@s.chibakoudai.jp} \\
			{\small Ryuichi UEDA, Chiba Institute of Technology} \\
			{\small Yasuo HAYASHIBARA, Chiba Institute of Technology}             \\
	\end{tabular}
}
\makeatother

\abstract{ \small
	Papers submitted must be original, and previously unpublished. The responsibility for the contents of published articles rests solely with the authors and not the society. Copyright of the papers published belongs to the JSME (Japan Society of Mechanical Engineers). [Abstract: Times New Roman, 9pt, 100-150words]
}

\date{} % 日付を出力しない
\keywords{Robot, Manipulation,… (no more than five words) [Times New Roman, 9pt]}

\maketitle
\thispagestyle{empty}
\pagestyle{empty}

\small
\section{緒言}%===========================
%本文:明朝体・9pt(欧文Times New Roman, 9pt)、文字間隔は1行26文字程度、行間隔は4.2mm程度にして下さい。
自律移動ロボットが実環境で走行することを考えると,変化する目的地や経路に対して即座に最適な行動計画が必要となる.価値反復を移動ロボットのナビゲーションに適用するとロボットの様々な状態に対して最適な行動を計算することができる.
%\subsection{論文作成に関する注意事項を以下に示します。(中見出し:ゴシック体・9pt・強調文字・左寄せ)}%-----------


%※ ただし、PDFファイルの容量は2MB以下、論文のページ数は2頁以上4頁以下とします。なお、印刷原稿の提出は不要ですので、郵送しないで下さい。

%※ 講演番号、講演会名、ページ番号は記載しないようにして下さい。
\section{実験}%===========================

\subsection{実験の環境}

\begin{table}[hbtp]
  \caption{experimental environment}
  \label{table:data_type}
  \centering
  \begin{tabular}{lcr}
    \hline
    CPU & Core™  \\
    GPU &  \\
    Ubuntu & 20.04 \\
  
    \hline
  \end{tabular}
\end{table}

\subsection{実験の方法}

\subsection{実験結果}

\section{結論}%===========================




\footnotesize
\begin{thebibliography}{99}

	\bibitem{Shinjuku98}
	新宿大五朗,渋谷次郎,東京 学,``キャスティングマニピュレーションに関する研究(第1報,可変長の紐状柔軟リンクを有するマニピュレータの提案とそのスイング制御法)'',{\it 機論C編}, vol.64-626, pp.3854--3861, 1998.

	\bibitem{Shinjuku99}
	Shinjuku, D., Shibuya, J. and Tokyo, M., ``Swing Motion Control of Casting Manipulation,'' {\it IEEE Control Systems}, vol.19-4, pp.56--64, 1999.

\end{thebibliography}

\normalsize
\end{document}
